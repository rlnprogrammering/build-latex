\section{Methodology}
\label{methodology}
This section explains the methods used in our research and development process, focusing on why and how we chose these approaches. We adopted an exploratory research method and the diary method, guided by the early-stage nature of the Beckn Protocol and the need for a flexible, reflective process \citep{winston2019exploratory}.

\subsection{Exploratory Research Approach}
We selected an exploratory research method because the Beckn Protocol is a new field with limited academic resources \citep{winston2019exploratory}. According to \citet{winston2019exploratory}, this approach is ideal for emerging topics, as it allows researchers to explore ideas flexibly before deciding what to measure. This method suited our project, enabling us to adjust our focus as we learned more. For example, we shifted from planning a decentralized food delivery network to creating an open-source automation tool for the Beckn-ONIX infrastructure. The exploratory approach allowed us to adapt to challenges like incomplete documentation and technical issues, ensuring our project remained practical and relevant.

\subsection{Diary Method}
To guide our reflection, we used the diary method, inspired by \citet{Naur1983}, who recommends keeping a log to track insights and progress in software development. We maintained a shared logbook to record reflections, technical milestones, and challenges - the logbook contents can be found in appendix \ref{logbook_appendix}. This method was chosen to help us stay organized and reflective, allowing us to document our learning process and revisit decisions as needed. Regular logbook entries, combined with team discussions, kept us aligned and supported our ability to adapt based on new insights. The diary method was especially useful in an exploratory project, as it helped us organize our thoughts and justify our changes \citep{Naur1983}.

\subsection{Development Process}
Our development process used an informal iterative approach, which worked well with our exploratory research. This method let us tackle technical challenges step-by-step, adjusting our goals as we gained new understanding. We applied Infrastructure as Code (IaC) principles to build the automation script, ensuring it was reliable and easy to use (Section~\ref{sectionIaC}). We chose shell scripting to match the Beckn-ONIX tools, making our work compatible with the existing system \citep{beck_onix_github}. Collaboration tools and regular work sessions supported teamwork and problem-solving.

\subsection{Rationale for Methodological Choices}
The exploratory method and diary method were ideal for a project in a new and evolving field like the Beckn Protocol. The exploratory approach gave us the flexibility to adapt to new information, while the diary method provided a structured way to reflect and track our progress, following \citet{Naur1983}’s advice for thoughtful development. The iterative development process ensured our automation tool was both technically sound and responsive to the protocol’s challenges. These methods allowed us to work systematically while staying open to change, aligning with our project’s goals.