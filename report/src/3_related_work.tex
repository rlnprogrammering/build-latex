\section{Related Work}
\label{related_work}
\subsection{Continuous Software Engineering}
\label{continuous_software_engineering}
Continuous Software Engineering (CSE) integrates practices such as Continuous Integration (CI), Continuous Delivery (CD), and Continuous Deployment to optimize the software development lifecycle through automation, frequent iterations, and rapid feedback loops. As defined by Fitzgerald and Stol, CSE is ``a holistic approach integrating cultural, organizational, and technological aspects to enable frequent releases of high-quality software'' \citep{fitzgerald2015continuous}. This section explores the theoretical foundations of CSE, its alignment with Lean and Agile methodologies, common implementation challenges, and proposed improvements to enhance its effectiveness. It concludes by linking CSE to DevOps practices, setting the stage for their application in a Beckn specific context discussed in later sections.

\subsubsection{Conceptual Foundations}
\label{cse_conceptual_foundations}
CSE streamlines development, testing, and deployment into a cohesive, iterative process. Continuous Integration (CI) involves frequently integrating code changes, with each change verified by automated builds and tests to detect integration errors quickly. Continuous Delivery (CD) ensures that software can be released to production at any time through automated processes, requiring only a manual trigger for deployment. Continuous Deployment further automates this, deploying every change that passes automated tests to production automatically. These practices enable rapid delivery while maintaining quality.

CSE aligns with the Build-Measure-Learn cycle from Lean Startup methodology, which Ries describes as a process to ``build a product, measure its performance with data, and learn from feedback to refine it'' \citep{ries2011lean}. This iterative approach ensures development responds to user needs. Additionally, CSE embodies Lean principles, which \citet{poppendieck2003lean} define as ``eliminating waste, amplifying learning, and delivering value as quickly as possible''. By automating repetitive tasks and fostering modularity, CSE minimizes non-value-adding activities, such as manual testing or environment setup. Agile principles, emphasizing iterative development and collaboration, further underpin CSE by promoting adaptability and cross-functional teamwork.

\subsubsection{Challenges in Applying CSE}\label{CSE_challenges}
Implementing continuous practices in complex, software-intensive systems presents significant challenges. One key issue is the complexity of setting up and maintaining development and testing environments. Shahin et al. highlight that simulating or accessing production-like environments, especially in systems with interdependent components or involving databases, can be resource-intensive and constitute a major part of the effort \citep{shahin2017continuous}. These environment-related difficulties hinder effective testing and slow down feedback loops, particularly when architectures are highly coupled or when dependencies span across teams or hardware systems. This hinders individual development, integration, and deployment, resulting increased technical debt.

Effective team collaboration in distributed or modular workflows is another challenge. Developing complex software with distributed teams requires ensuring tight integration and building in quality from the beginning. This necessitates practices like continuous integration and continuous testing, supported by mechanisms such as Poka Yoke, which rely on standardization to quickly detect and prevent errors \citep{fitzgerald2015continuous}. Furthermore, the fast-paced nature of CSE requires solid feedback mechanisms. Inadequate automated testing or monitoring can result in defects reaching production, undermining the benefits of frequent releases. These challenges impede the efficiency and scalability of CSE.

\subsubsection{Proposed Improvements} \label{CSE_improvements}
To overcome these challenges, several strategies can enhance CSE implementation. First, automating environment setup and configuration reduces overhead and enables component isolation. Tools that provision pre-configured, production-like environments allow developers to test modules independently, aligning with CSE’s focus on modularity. The Shift Left strategy, which involves moving testing and validation earlier in the development process to address issues like technical debt proactively, is particularly effective here. \citet{fitzgerald2015continuous} describe Shift Left as integrating testing and validation early in the development process to address technical debt proactively, ensuring defects are caught before they propagate.

Second, standardized tools and processes improve collaboration and consistency. Containerization technologies, such as Docker, create reproducible environments and streamline deployment, supporting parallel development. Comprehensive automated testing suites—covering unit, integration, and end-to-end tests—ensure timely and actionable feedback, reducing defect risks. \citet{fitzgerald2015continuous} emphasize that automation of testing and deployment processes is critical to sustaining fast-paced iteration cycles.

Finally, fostering a culture of continuous improvement is essential. Regular evaluation and refinement of processes help identify bottlenecks and optimize automation. This aligns with Lean’s focus on waste elimination and Agile’s emphasis on adaptability, creating a resilient development pipeline. \citet{poppendieck2003lean} stress that continuous improvement requires a commitment to learning and adapting processes based on empirical feedback.

CSE principles provide a foundation for DevOps, which extends automation and collaboration to deployment and operations. DevOps builds on CSE through automated deployments, continuous monitoring, and cross-functional collaboration. In later sections, we explore how these theoretical concepts apply to specific frameworks, such as decentralized digital ecosystems, to address their unique challenges and enhance development efficiency.

\subsection{Domain-Driven Design} \label{domain_driven_design}
Domain-Driven Design (DDD) is a strategic approach to software development that focuses on modeling software based on the complex needs of the business domain it serves. As described by \citet{vernon2013DDD}, DDD is not just about software architecture, but about bridging the gap between domain experts and software developers by fostering deep collaboration and shared understanding among teams.

Two of the core pillars that enable DDD’s strength and effectiveness are the \emph{Ubiquitous Language} and the \emph{Bounded Context}. The following descriptions are based on Vernon’s interpretation of DDD principles in implementing Domain-Driven Design \citep{vernon2013DDD}.

\emph{Ubiquitous Language} is a common, rigorous language developed by the team, including both developers and domain experts, that is based on the domain model. This language is used consistently in code, design documents, and conversations. The idea is that the model isn't just implemented in code; it becomes a shared way of thinking and communicating about the domain. When the team speaks the same language, misunderstandings are reduced, and the code reflects the real-world problems it intends to solve more clearly and accurately. 

\emph{Bounded Context} defines a logical boundary within which a specific domain model is defined and applicable. Within a bounded context, the terms in the ubiquitous language have precise meanings. This allows different parts of a larger system to use similar terms with different interpretations, without introducing ambiguity. For example, the term "Transaction" might refer to a financial payment in one context and a user action or event in another. Clearly describing these contexts avoids conceptual confusion and encourages the development of models that are cohesive and focused.

In practice, these two pillars work hand in hand: the Ubiquitous Language thrives within a clearly defined Bounded Context. Together, they help teams manage complexity, align software with business goals, and produce systems that are flexible and deeply rooted in the domain.

Ultimately, DDD helps model complex domains in the simplest way possible and addresses various architectural and design concerns, including supporting scalability and distributed computing.

\subsection{DevOps}
\label{devops}
\subsubsection{DevOps for Distributed Systems}
\label{devops_distributed}
Distributed systems are fundamentally different and have inherent complexity that developers must understand, which presents a whole new set of challenges for developers working with distributed systems. These challenges often involve issues that are difficult to identify and test in a code environment, such as a developer’s local machine, due to the increased complexity of interconnected systems. One widely recognized set of misconceptions that highlights these difficulties is known as the "Fallacies of Distributed Computing", a list of flawed assumptions developers often make when designing distributed systems. These fallacies were originally proposed by Peter Deutsch of Sun Microsystems in 1994 and were later expanded to eight by others at Sun, including James Gosling:
\begin{enumerate}
    \item The network is reliable
    \item Latency is zero
    \item Bandwith is infinite
    \item The network is secure
    \item Topology doesn't change
    \item There is one administrator
    \item Transport cost is zero
    \item The network is homogeneous
\end{enumerate}

These fallacies remain highly relevant today and are well-documented in both industry and academic literature \citep{rotemgaloz_fallacies}.

As previously discussed, distributed systems are fundamentally complex, often making it difficult for one person to see the whole system. One of the core DevOps principles aims to counter this: Feedback, which enables us to create safer systems of work. This includes the use of pervasive production telemetry, meaning continuous monitoring and instrumentation embedded throughout both the code and production environments. Such telemetry helps detect anomalies, failures, or unexpected behavior early, allowing issues to be resolved quickly before they evolve into further technical debt. By enabling fast feedback loops, teams can continuously validate that the system behaves as intended, ultimately ensuring the software remains resilient, reliable, and capable of delivering sustained value to its users \citep{kim2016devops}.

The traditional silos and handoffs in IT include coordination problems frequently seen in distributed systems, where knowledge is spread across administrators. This challenge is especially evident in open-source software, where the creators are often unknown, leaving developers to rely entirely on documentation and community knowledge sharing. DevOps promotes breaking the conflict between development and operations, by encouraging teams to work together, in some cases embedding operations expertise directly into development teams through the use of shared tools and practices. The shared responsibility not only improves collaboration, but also enforces it as it helps address production issues faster. As a result, problems are resolved more quickly, and services are built to be more maintainable, rather than leaving operations teams to deal with unstable or unsupported systems.


\subsection{Infrastructure as Code}\label{sectionIaC}
Infrastructure as Code (IaC) is a core enabler of modern DevOps practices, enabling teams to define, provision, and manage infrastructure using code rather than manual processes. As \cite{morris_2020} describes, \textit{“DevOps is a movement to reduce barriers and friction between organizational silos like development and operations … Technology and engineering practices like Infrastructure as Code should be used to support efforts to bridge the gaps and improve collaboration.”} IaC supports this goal by introducing automation, repeatability, and transparency into infrastructure management, fostering shared responsibility across teams.

\subsubsection{Infrastructure as Code in Domain-Oriented Architectures}
As previously mentioned, distributed systems present significant coordination challenges due to their inherent complexity and the difficulty of maintaining shared understanding across services \ref{devops_distributed}. In such systems, Domain-Driven Design (DDD) addresses this by advocating for clear boundaries through Bounded Contexts and a shared conceptual model via the Ubiquitous Language \ref{domain_driven_design}. These principles help teams reason about and align their work around well-defined domains. Infrastructure as Code (IaC) reinforces these same principles by enabling isolated, reproducible infrastructure definitions that map cleanly to each bounded context. This allows teams to define and provision infrastructure independently, using consistent automation that aligns with the modular, loosely coupled system architecture promoted by DDD. Furthermore, IaC minimizes configuration drift and ensures predictable deployments across environments, which is essential for maintaining the integrity of distributed systems \citep{morris_2020}.

\subsubsection{Modularization and Single Responsibility} \label{modularization}
Modularization in Infrastructure as Code (IaC) refers to structuring infrastructure into self-contained components, each with a single, well-defined responsibility. This reflects the \textit{separation of concerns} principle in software engineering, which advocates dividing a system into distinct parts that address specific functionality to reduce complexity and improve maintainability \citep{dijkstra1982}. By isolating concerns, each module can be developed, tested, and evolved independently, leading to more robust and adaptable systems.

For example, infrastructure modules might individually manage aspects such as container orchestration, codebase retrieval, service deployment, or environment-specific configuration. This separation allows teams to modify one area without impacting others, facilitating collaborative development, independent iteration, and simpler maintenance as distributed systems grow (\citep{morris_2020})

The single responsibility principle reinforces this approach by ensuring each module or script performs one focused task. This improves testability, traceability, and reduces the risk of unintended side effects, especially in dynamic infrastructure environments where requirements frequently change. Together, modularization and single responsibility enable IaC systems to remain resilient, scalable, and easier to reason about over time \citep{morris_2020}.

\subsubsection{Parameterization and External Configuration} \label{parameterization_external_config}
Parameterization and external configuration are key practices in Infrastructure as Code (IaC) that promote reusability, consistency, and maintainability across environments. Parameterization involves abstracting environment-specific values, such as IP addresses, hostnames, or resource limits, into inputs, rather than hardcoding them into scripts. This enables teams to use the same infrastructure definitions across different stages (e.g., development, staging, production) simply by supplying different parameters \citep{morris_2020}. Common mechanisms for this include environment variable files, configuration templates, or command-line arguments.

By decoupling logic from configuration, parameterization increases portability, reduces the risk of configuration drift, and simplifies onboarding, since changes to infrastructure behavior can be made without modifying code. This is especially valuable in distributed systems, where maintaining consistency across multiple services and environments is essential for stability and predictability.

Complementing parameterization, external configuration involves separating configuration data from execution logic into dedicated, version-controlled files. This not only supports cleaner codebases, but also enhances security, particularly when used with secret management tools or encrypted files to handle sensitive values \citep{morris_2020}. Together, these practices support repeatable deployments and enable infrastructure to be more adaptable to evolving requirements, all while maintaining a high standard of maintainability and clarity.


\subsubsection{Executable Documentation} \label{IaCDocumentation}
A core benefit of Infrastructure as Code is its function as executable documentation. A concept emphasized throughout Infrastructure as Code \citep{morris_2020}. Traditional infrastructure documentation often takes the form of static artifacts such as README files, wikis, or internal notes. These documents are frequently out of sync with the actual implementation, particularly in fast-moving or distributed teams. In contrast, IaC allows infrastructure definitions to act as both implementation and documentation simultaneously. The IaC tools or scripts encode the logic, sequencing, dependencies, and configuration of infrastructure directly into version-controlled code. This dual role ensures consistency, and reduces the need of cross-referencing documentation with actual system behavior. According to Morris, this approach creates “a single source of truth,” enabling any team member to understand and reproduce the environment by reading and running the code itself. Furthermore, modular and parameterized structures reinforce this clarity, allowing each component to document its own responsibility within the system. 

\subsubsection{Open Source}
\label{open_source}
Open source is a key enabler of Infrastructure as Code (IaC), promoting collaboration, transparency, and reproducibility in infrastructure management. By developing our automated Beckn-ONIX setup scripts as an open-source project, we align with the Beckn Protocol’s ethosthe open ecosystem, fostering accessibility for developers of varying expertise. Hosted on a public GitHub repository, our scripts allow anyone to study, modify, and distribute the code, supporting the Beckn community’s mission to democratize digital commerce \citep{stallman1985gnu}. This open-source approach, detailed further in Section~\ref{product}, ensures our IaC tool is a shared, improvable resource, reducing technical barriers and enhancing the scalability of Beckn networks.