\section{Discussion}
\label{discussion}
\subsection{Infrastructure as Code for Beckn Networks}
\label{IaC_discussion}
Although the automation script developed in this project was initially driven out of spite of how tedious it was setting up a Beckn production environment, it has evolved into much more than just a time-saver. By codifying the network setup process and making it reproducible, the tool lays a foundation for both quick experimentation and long-term maintainability. Qualities that benefit not just newcomers but also experienced teams operating within the Beckn ecosystem.

\subsubsection{Lowering the Barrier for New Developers}
For new developers, the automation script fulfills the original promise advertised by the Beckn-ONIX project: that setting up a functioning Beckn network should take minutes, not days. While Beckn-ONIX advertises quick deployment (section~\ref{beckn_onix}), our early experience proved that the reality was far from plug-and-play. Our tool bridges this gap between developers and operation experts as mentioned in (section~\ref{sectionIaC}), delivering on that original promise and enabling developers to begin building Beckn Application Platforms (BAPs), Beckn Provider Platforms (BPPs), or even domain-specific languages almost immediately.
This shift from setup to development time is important. Rather than spending days troubleshooting environment issues, developers can now focus on understanding the Beckn architecture, experimenting with use cases, or prototyping domain-specific features, exactly as we had originally hoped to do. In effect, the script reclaims valuable time and encourages it toward useful innovation instead. It lowers the entry threshold and makes Beckn more accessible to smaller teams, students, open-source contributors, or startups with limited DevOps capacity.

\subsubsection{Supporting DevOps and CI/CD for Established Teams}
While the automation script’s value for beginners is very clear, its potential for established teams is just as exciting, though less obvious at first glance. In production environments, where stability and predictability are essential, the script can serve as a tool for replicating infrastructure. For example, teams can use it to spin up isolated, production-like test environments to validate new BAPs or BPPs before integrating them into the live network. This practice aligns with DevOps principles such as continuous testing (section~\ref{cse_conceptual_foundations}) helping to catch integration issues early and reduce future technical debt.

Because each component setup is handled by a dedicated, modular child script, the orchestrator can also be tailored to specific tasks, like deploying just a new BPP into an existing registry and gateway setup. This makes the tool well-suited to incremental changes, or internal sandboxing. By integrating it into CI/CD pipelines, teams can fully automate these processes, improving deployment reliability for existing networks.

\subsubsection{Improved documentation}
Improved documentation is one of the few big advantages of Infrastructure as Code, as highlighted in (section~\ref{IaCDocumentation}) , and our experience with Beckn-ONIX reinforces its importance. In traditional software projects, infrastructure is often documented separately from its implementation, either through README files, wikis, or internal notes. These forms of documentation tend to be inconsistent or quickly outdated, particularly in fast-evolving or poorly maintained projects. Beckn-ONIX showcases this challenge: although it promotes itself as a rapid deployment solution, the official documentation is missing steps, is inconsistent, and at times misleading. Critical setup steps such as SSL certificate provisioning, webhook integration, and hardware requirements are either vaguely documented or entirely omitted. This forced us to rely on trial and error to get a working environment.

By contrast, our orchestrator script effectively works as the documentation. It defines not only what needs to be done but also how and in what sequence. The logic is explicit, executable, and self-validating. Rather than describing abstract procedures in small steps, it enforces them consistently across desired environments. This makes it far more reliable than static documentation, especially in open ecosystems where developers vary widely in background and tooling.

Moreover, the modular and parameterized structure of the script makes its behavior transparent. Each task, whether provisioning Docker, configuring NGINX, or registering subdomains, is encapsulated in dedicated, single-responsibility scripts. These scripts are readable, reusable, and logically grouped, serving as living documentation for both how the network operates and how it is expected to be maintained. The .env file centralizes all variables, acting as both a configuration hub and a reference point for system topology.

This level of clarity is especially beneficial in complex system infrastructures like Beckn, where multiple independent users need to understand and replicate infrastructure reliably. While Beckn-ONIX’s promise of simplicity sounds good in practice, its current documentation does not meet the level of detail and reliability required in production settings. Our tool directly addresses this gap by translating infrastructure knowledge into code that both configures and explains the network, aligning with Morris' (2021) view of IaC as executable documentation as described in (section~\ref{IaCDocumentation}).

By combining automation with clarity, the tool transforms the Beckn setup into something both reproducible and maintainable. It directly addresses the documentation challenges we encountered and sets a stronger foundation for others to build on.

\subsubsection{Open-Source Impact}
The open-source nature of our orchestrator script, detailed in Section~\ref{open_source_contribution}, amplifies its impact by enabling global collaboration within the Beckn community. By sharing our automation tool publicly, we reduce setup barriers, encouraging contributions from diverse developers and supporting the creation of innovative, community-driven Beckn networks, such as those envisioned for Copenhagen’s food delivery market.

\subsection{Future Work}
\subsubsection{Migrating to other tools}
Our project utilized a Bash-based orchestrator shell script to automate the setup of a Beckn production network, as detailed in section~\ref{product_orchestrator} This approach, while effective in achieving a significant reduction in deployment time (approximately 99\% compared to manual setup, as noted in section~\ref{tech_contribution_and_impact}), relies on shell scripting conventions aligned with the Beckn-ONIX community's existing tools. However, shell scripts, despite their portability and minimal dependencies, have limitations in terms of scalability, error handling, and maintainability for complex distributed systems. A more robust tool, such as Ansible, could enhance the automation process.

Ansible, an open-source automation platform, is designed specifically for configuring and managing distributed systems. Unlike our Bash orchestrator, which requires manual intervention to handle errors or adapt to varying server configurations, Ansible uses a declarative YAML-based syntax to define infrastructure states, ensuring idempotent executions and consistent outcomes across diverse environments \citep{morris_2020}. For instance, Ansible's built-in modules for managing Docker, NGINX configurations, and SSL certificates could replace our custom child scripts (e.g., \texttt{setup\_docker.sh}, \texttt{generate\_all\_motifs.sh}), reducing the need for manual error handling and improving reliability. Additionally, Ansible's inventory management would streamline the parameterization of server-specific variables, currently managed through a \texttt{.env} file, by dynamically mapping roles to servers without requiring manual SSH configurations.

Transitioning to Ansible could also address some of the limitations identified in Section~\ref{third_sprint}, such as the lack of idempotency in our scripts. Our current orchestrator does not inherently skip completed tasks on re-runs, which can lead to errors or redundant operations. Ansible's idempotent design ensures that only necessary changes are applied, enhancing the script's robustness. Furthermore, Ansible's extensive logging and error-reporting capabilities would provide clearer feedback to users compared to our script's basic success/failure reporting, improving the user experience for developers with varying expertise levels.

While our Bash-based solution meets the immediate needs of the Beckn community by maintaining compatibility with existing tools, adopting Ansible could elevate the automation to industry best practices. This shift would require refactoring our scripts into Ansible playbooks, a process that could be incrementally implemented to maintain compatibility with the current setup. Such an improvement would align with the DevOps principle of continuous improvement, as discussed in Section~\ref{devops}, and better support the Beckn ecosystem's goal of lowering technical barriers for diverse users.

\subsubsection{Adaptation to Different Development Requirements}
The modularity of our orchestrator script, as described in Section~\ref{modularity_single_responsibility}, allows developers to customize the deployment by commenting out unwanted tasks in the \texttt{TASKS} array. For example, a developer needing only a Beckn Provider Platform (BPP) to connect to an existing network can exclude scripts related to the registry, gateway, or Buyer Application Platform (BAP). However, this approach requires manual code modifications, which can be error-prone and less intuitive for non-expert users. To enhance flexibility, the orchestrator could be improved by introducing command-line parameters to dynamically select which components to deploy, eliminating the need to edit the script directly.

For instance, a flag-based interface, such as \\
\texttt{./orchestrator.sh --components registry,gateway,bap}, could allow users to specify the desired components at runtime. This would leverage the existing modularity of our child scripts, as each task is already encapsulated with a single responsibility (e.g., \texttt{setup\_bpp.sh} for BPP deployment). Implementing this would involve updating the orchestrator to parse command-line arguments and filter the \texttt{TASKS} array accordingly. Such a feature would align with the parameterization principle outlined in section~\ref{parameterization_external_config}, extending it to runtime configuration and making the tool more user-friendly.

This enhancement would cater to diverse development contexts, as discussed in Section 6.2.1. For example, a company aiming to integrate a BPP into an existing Beckn network could deploy only the BPP component, while a researcher testing a full network could deploy all components. Additionally, this flexibility could support the creation of isolated test environments, as noted in Section 6.1, by allowing developers to spin up specific components on internal or cloud servers without deploying the entire network. This adaptability would be particularly valuable for open-source contributors, who often work in varied environments and require tailored setups.

To further improve, the orchestrator could incorporate validation checks to ensure that selected components form a functional network. For instance, deploying a BPP without a gateway might be invalid unless the BPP is connecting to an external network. Adding such logic would enhance the script's robustness and prevent misconfigurations. These improvements would build on the DevOps feedback principle Section~\ref{devops_distributed}, providing immediate validation to users and reducing setup errors.

By introducing runtime parameters and validation, the orchestrator would become a more versatile tool, supporting a wider range of use cases while maintaining the simplicity and accessibility that are central to the Beckn ecosystem's open-source ethos. These enhancements would empower developers to focus on building innovative applications, such as the decentralized food delivery network we initially envisioned Section~\ref{project_description}, by streamlining the setup process for both full and partial network deployments.