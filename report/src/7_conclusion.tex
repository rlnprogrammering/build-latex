\newpage
\section{Conclusion}
\label{conclusion}
%Not sure if below needs to actually be subsections, but that's what we gotta write about at least
% \subsection{Research question}
% \subsection{Summary of research}
% \subsection{Summary of discussion}

Our project initially aimed to build a decentralized food delivery network for Copenhagen using the Beckn Protocol. However, the complex and error-prone setup process of Beckn-ONIX prompted us to shift focus to a more urgent challenge within the Beckn community. We addressed the research question, "How can we improve and optimize the setup of a Beckn production network to maximize usability and minimize deployment time?" by developing an Infrastructure as Code tool. This orchestrator script automates the deployment of a fully functional Beckn network, reducing setup time from days to roughly 15 minutes. User testing with participants of varying expertise confirmed a 99\% time reduction and demonstrated the script’s accessibility, enabling developers to deploy a network with minimal effort.

The script, hosted on a public GitHub repository (https://github.com/Bachelors-Project-frlr-raln/beckn-IaC/tree/v1.0.2), includes a detailed README that fills critical gaps in Beckn-ONIX’s documentation. It serves as both an executable solution and a clear guide, embodying the open-source values of the Beckn Protocol and the spirit of shared empowerment championed by the GNU Manifesto (Stallman, 1985). By simplifying the setup process, our tool allows developers to focus on creating innovative applications, such as the equitable food delivery platform we originally envisioned, rather than wrestling with technical hurdles.

Future enhancements, such as adopting Ansible for greater scalability or adding command-line parameters for flexible deployments, could further strengthen the script. By sharing our work openly, we aim to support the Beckn community in building inclusive digital marketplaces. Our contribution not only streamlines a critical process but also fosters collaboration, laying a foundation for others to innovate and create accessible, community-driven networks in Copenhagen and beyond.