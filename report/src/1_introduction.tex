\newpage
\newpage
\section{Introduction}
\label{introduction}
\subsection{Motivation}
\label{motivation}
When we started looking into possible research topics for our Bachelor's project back in December, we both agreed that we would like to \emph{create something.} We had many ideas, ranging from an updated platform for reporting scores and calulating ELO ratings in sports like tennis, badminton and tabletennis, all the way to a stock trading bot. After having talked with multiple different supervisors, we talked with Yvonne, who was very enthusiastic about the Beckn Protocol and explained it on a high level. After our talk we did some research into it, and agreed that it sounded promising and that we could align it with our motivation to \emph{create something}. We could see several possible implementations that could have a positive impact in our, as well as others, daily lives in Copenhagen, such as for example mobility, where we considered creating a unified app for all the different rental bikes and scooters that are currently scattered across many different platforms. Another idea we had, which is the one we ultimately went with, was to create a decentralized network for food ordering, where there was no middle man. Something that the Beckn protocol would be well fitted to solve.

The food ordering market in Copenhagen is currently dominated by companies like Wolt and Just Eat, where we see them biking around the city in their blue and orange uniforms. However, one issue that limits our usage of these specific platforms, is the price difference compared to just picking up the food ourselves. Wolt for example, charges restaurants a 30\% commission rate, which is then passed on to the customers. A durum at our local kebab shop costs 55 on their own webpage, and 70 on Wolt. Then you are also charged delivery ranging anywhere from 9-89 DKK, as well as a service fee of up to 19 DKK. Delivery drivers meanwhile only earn 160 DKK on average \citep{wolt_faqs}. These price increases are noticeable to consumers like ourselves, and all stem from the fact that a centralized platform facilitating the contact also needs to turn a profit. By creating a decentralized network utilizing the Beckn Protocol, allowing peer-to-peer communication and transactions, this is no longer the case. It would foster a more equitable and fair solution, with reduced entry barriers for smaller restaurants. Furthermore it would most likely increase the volume of orders, creating more job opportunities for delivery drivers. Our concerns regarding high commission rates affecting restaurants is reflected in \citet{seghezzi_mangiaracina_2021} while \citet{li_wang_2024} notes that regulatory efforts to cap such commissions negatively impact order volumes for small businesses, highlighting the need for alternative solutions.

\subsection{Project description}
\label{project_description}
While we started out with the goal of creating a decentralized food delivery network, tailored to the needs of Copenhagen, we had to change the direction of our project. Based on the research we had done early on, and Beckn's own documentation highlighting rapid deployment through Beckn-ONIX (Open Network In a Box), we believed that it was feasible to create a functioning network and onboard local restaurants. We quickly found out, that it unfortunately was not the case. When we tried deploying the Beckn Network, we encountered countless errors, due to various factors. By the time we had solved all these problems, we found that a more meaningful contribution to the Beckn Community would be to share our findings condensed down to an automated and improved setup script, that would allow future developers to focus on creating innovative networks, rather than with setup-struggles. 
\subsection{Research question}
\label{research_question}
Originally, our research question was: “How can we create a more equitable food delivery solution utilizing the Beckn Protocol to decentralize order flow, specifically in the Copenhagen area?” The deployment challenges we encountered changed our focus to “How can we improve and optimize the setup of a Beckn production network to maximize usability and minimize deployment time?” This change addresses a critical barrier within the Beckn Community, and our contribution supports the Beckn Protocols broader adaption.

\subsection{Outline}
\label{outline}
Throughout the remainder of this report, we document our research, the pivot to automation, and our contribution to the Beckn ecosystem. Section~\ref{beckn_protocol} introduces the Beckn Protocol, detailing its architecture, ecosystem, and the challenges encountered with Beckn-ONIX deployment. Section~\ref{related_work} reviews related work, covering Continuous Software Engineering, Domain-Driven Design, DevOps, and Infrastructure as Code, which inform our automation approach. Section~\ref{methodology} outlines our methodology, including the exploratory research approach, diary method, and iterative development process. Section~\ref{presentation_of_research} presents our research findings, describing the three-sprint development process, the orchestrator script, and its evaluation through user tests. Section~\ref{discussion} discusses the implications of our work for distributed systems, compares our approach in different development contexts, and suggests future improvements. Finally, Section~\ref{conclusion} concludes with a summary of our contributions, revisiting the research question and key findings.